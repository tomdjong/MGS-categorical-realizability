\chp{PCA}

% ==== 2.11 ====

\exercise{properties-of-lambda-pca}

Each of the properties (i)--(iii) are proved by induction on the structure of the
body term \(t\).

\begin{enumerate}[(i)]
  \item
    The claim holds when \(t = x\),
    when \(t = y\) is a variable distinct from \(x\),
    and when \(t = \pca a \in \AA\).

    Finally, by the induction hypothesis the set of variables in the term
    \[ \lambdapca{x}{(t_1\,t_2)} =
      \scomb\,(\lambdapca{x}{t_1})\,(\lambdapca{x}{t_2}) \]
    is exactly
    \((\Var{t_1} \setminus x) \cup (\Var{t_2} \setminus x)\),
    where \(\Var t\) denotes the set of variables in the term \(t\).
    The result in this case now follows since
    \[ \Var{t_1\,t_2} \setminus x =
      (\Var{t_1} \setminus x) \cup (\Var{t_2} \setminus x). \]

  \item
    The term \(\lambdapca{x}{t}\) is defined when \(t\) is a variable or an
    element of \(\AA\) since, by \cref{def:pca}, \(\kcomb \pca a\) and
    \(\scomb \pca a \pca b\) are defined for any \(\pca a, \pca b \in \AA\).

    Finally, if \(\lambdapca{x}{t_1}\) and \(\lambdapca{x}{t_2}\) are
    defined then any substitution into the variables of
    \[ \lambdapca{x}{(t_1\,t_2)} =
      \scomb\,(\lambdapca{x}{t_1})\,(\lambdapca{x}{t_2}) \]
    yields an element of \(\AA\) of the form \(\scomb\,\pca a\,\pca b\)
    for some $\pca a, \pca b \in \AA$.

  \item
    By straightforward computation in the case where \(t\) is not an
    application.
    When \(t = t_1\,t_2\),
    \begin{align*}
         & (\lambdapca{x}{(t_1\,t_2)}) \pca a \\
      =\ & \scomb \, (\lambdapca{x}{t_1}) \, (\lambdapca{x}{t_2}) \, \pca a \\
      \simeq\ & ((\lambdapca{x}{t_1}) \pca a) \, ((\lambdapca{x}{t_2}) \pca a) \\
      \simeq\ & (t_1[\pca a/x]) \, (t_2[\pca a/x])
        \qquad \text{(by the induction hypothesis)} \\
      \simeq\ & (t_1\,t_2)[\pca a/x].
    \end{align*}
\end{enumerate}


% ==== 2.14 ====

\exercise{pairing-projection}

\begin{enumerate}[(i)]
  \item
    \( \pcapair \pca a \pca b =
      (\lambdapca{xyz}{zxy}) \pca a \pca b =
      \lambdapca{z}{z \pca a \pca b}\)
    is defined by \cref{exer:properties-of-lambda-pca}.

  \item
    By computation, taking care to note throughout that all applications are
    defined in \(\AA\).
\end{enumerate}

% ==== 2.15 ====

\exercise{arithmetic}

A possible set of definitions is
\begin{align*}
    \pcaiszero & \defeq \pcafst \\
    \pcasucc & \defeq \pcapair \pcafalse \\
    \pcapred & \defeq \lambdapca{n}{\pcaif\,(\pcaiszero n)\,\numeral{0}\,(\pcasnd n)}
\end{align*}
(check that these satisfy the required equations).


% ==== 2.16 ====

\exercise{primitive-recursion}

\newcommand\spec{\pcacomb{spec}}

From the specification of  \(\pcarec\), we would like our definition to satisfy
the equation
\[ \pcarec \pca{a} \pca{f} \simeq
  \lambdapca{n}{\big(
    \pcaif \, (\pcaiszero n) \,
      \pca{a} \,
      \big(\pca{f} \, (\pcapred n) \, (\pcarec \pca{a} \pca{f} \, (\pcapred n))\big)
  \big)}
\]
for any \(\pca a, \pca f \in \AA\).

This suggests that the term \(\pcarec \pca{a} \pca{f}\) should be constructed as
a fixed point of the abstraction
\begin{equation} \label{eq:spec}
\lambdapca{r}{\big(
  \lambdapca{n}{\big(
    \pcaif \, (\pcaiszero n) \,
      \pca{a} \,
      \big(\pca{f} \, (\pcapred n) \, (r \, (\pcapred n))\big)
  \big)}
\big)},
\end{equation}
and so we might try to define
\[ \spec^\prime \defeq
\lambdapca{af}{\lambdapca{rn}{
  \pcaif \, (\pcaiszero n) \,
    a \,
    \big(f \, (\pcapred n) \, (r \, (\pcapred n))\big)
}}
\]
and
\[ \pcarec^\prime \defeq \lambdapca{af}{\pcaz \, (\spec^\prime a f)}. \]
However, this definition does not satisfy the requirement that
\(\pcarec^\prime \pca{a} \pca{f} \numeral{0}\) is always defined (expand the
definition and check!).

Instead, we tweak the abstraction \eqref{eq:spec} whose fixed point we take,
and define
\begin{align*}
\spec & \defeq
\lambdapca{af}{\lambdapca{rn}{
  \pcaif \, (\pcaiszero n) \, (\kcomb a) \, (\scomb f r)
    (\pcapred n)
}}, \\
\pcarec & \defeq \lambdapca{af}{\pcaz \, (\spec a f)}.
\end{align*}
We can then check (do so!) that the required equations are satisfied.


% ==== 2.17 ====

\exercise{nontrivial-pca}

\begin{description}

\item[(i) \(\Longrightarrow\) (ii):]
  Assuming \(\pcatrue \neq \pcafalse\), we show that
  \(\numeral{m} \neq \numeral{n}\) for all \(n \in \Nat\) and \(m < n\),
  by case distinction on \(n\) and then induction on \(m < n\).
  This is trivial for \(n = 0\).

  Assume that \(n = n^\prime + 1\).
  For \(m = 0\),
  \[ \pcaiszero \numeral{m} = \pcatrue \neq \pcafalse = \pcaiszero \numeral{n} \]
  and so \(\numeral m \neq \numeral n\).
  If \(m+1 < n\) then \(m < n^\prime\) and
  \[ \pcapred \numeral{m+1} = \numeral m
    \neq \numeral{n^\prime} = \pcapred \numeral{n} \]
  by induction, so \(\numeral{m+1} \neq \numeral n\).

\item[(ii) \(\Longrightarrow\) (iii):]
  Immediate.

\item[(iii) \(\Longrightarrow\) (i):]
  Because if \(\pcatrue = \pcafalse\) then
  \[ \pca a = \pcaif \pcatrue \pca{a} \pca{b} =
    \pcaif \pcafalse \pca{a} \pca{b} = \pca b \]
  for all \(\pca a, \pca b \in \AA\).

\end{description}