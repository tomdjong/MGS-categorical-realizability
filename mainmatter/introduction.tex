\chapter{Introduction}

\emph{Realizability} originates with Kleene~\cite{Kleene1945} in the context of
proof theory and sought to make a precise connection between intuitionistic
(constructive) mathematics and computability theory.
%
This led to an effective interpretation of intuitionistic number theory using
computability theory, somewhat reminiscent to---but predating!---the
Curry--Howard correspondence.

In the 1980's, Hyland introduced the \emph{effective topos}: a category whose
logic is governed by Kleene's realizability.
%
Intuitively, the effective topos presents us with an alternative world of
mathematics where---unlike in the category of sets and functions---``everything is
computable''.
%
Actually, the effective topos is an instance of a general class of categories
known as \emph{realizability toposes} as originally developed by Hyland,
Johnstone and Pitts~\cite{HJP1980,Pitts1981}.

Every realizability toposes is parametrized by an abstract model of computation,
known as a \emph{partial combinatory algebra (pca)}. Classical computability
theory based on partial Turing computable functions on natural numbers gives a
pca known as \emph{Kleene's first model} and the resulting realizability topos
is Hyland's effective topos.
%
While Hyland's topos is perhaps the best understand example, many other choices
of pcas are possible with interesting realizability toposes as a
result~\cite{vanOosten2008}.

A beautiful aspect of Hyland's discovery is that it enables us to study the
interplay between category theory, logic and computability theory.
%
A typical application of realizability categories is to provide semantics to
(polymorphic) type theories like System~F (which has no direct set-theoretic
semantics~\cite{Reynolds1984}); see also \cref{sec:further-reading}.
%
Realizability has also found practical application in the form of extracting
programs from mathematical proofs, see e.g.~\cite{Minlog}.

\section{Aims}
We hope that these notes provide a self-contained and accessible introduction to
the categorical aspects of realizability for graduate students in theoretical
computer science.
%
More generally, we hope the reader will appreciate these notes as a testament of
the deep connections between category theory, logic and computability theory.

The notion of a realizability topos is fairly involved and for this reason we
focus on the simpler categories of \emph{assemblies} instead (although we
include a brief epilogue on realizability toposes).
%
While the category of assemblies lacks some features of a topos it provides more
than enough structure for our present purposes. Besides, one should arguably
first have a good grasp on the assemblies before studying realizability toposes.

\paragraph{Outline of these notes}
\begin{itemize}
\item \cref{chap:PCA} introduces partial combinatory algebra (pcas) as
  abstract models of computation and illustrate them with various examples.
  %
  Some familiarity with basic computability theory and topology is
  helpful, but certainly not required.
\item \cref{chap:assemblies} describes the category of assemblies and assembly maps
  over a fixed but arbitrary pca.
  %
  Intuitively, an assembly is a set together with computability data and an
  assembly map is a function of sets that is computable. Here, the notion of
  computability is prescribed by the pca.

  We explore the categorical structure of this category and some familiarity
  with basic category theory---say (co)limits and adjunctions---is necessary in
  most places, although I am hopeful that those unfamiliar with category theory
  can still benefit from this chapter and are perhaps inspired to learn some
  category theory.
\item
  Finally, in \cref{chap:logic}, we turn to the logical aspects of the category
  of assemblies.
  %
  We spell out the realizability interpretation of first order logic that the
  assemblies give rise to.
  %
  We illustrate the connections between category theory, logic and computability
  theory by studying certain \emph{realizability predicates} from these three
  perspectives. For example, the semidecidable predicates can be characterized
  (1)~categorically, as pullbacks of a certain two-element assembly;
  (2)~computably, as computably enumerable subsets; and (3)~logically, as those
  predicates which are presented by a binary sequence.
  %
  Finally, we exploit these connections by describing a simple result from
  \emph{synthetic computability theory}.
\end{itemize}


\section{Exercises}

The exercises are interspersed in the text, but each chapter ends with a list of
its exercises for reference.
%
There are \total{allexercises} exercises in total.

\section{References}

In preparing these notes I have mainly used the standard
textbook~\cite{vanOosten2008} by van Oosten, as well as Bauer's excellent
lecture notes~\cite{Bauer2023}.
%
In a few places, e.g.\ \cref{base-change-adjoints}, I have also consulted
Streicher's notes~\cite{Streicher2018}.
%
The presentation of logic in the category of assemblies using realizability
predicates owes a lot to Bauer's treatment, although I turned to
\cite[Section~3.2.7]{vanOosten2008} for
\cref{exer:Sigma-in-Kleene-1,exer:Sigma-to-N-is-CE,exer:Sigma-ce-subsets,exer:Rice-consequence},
and included some additions in the form of
\cref{sec:revisiting-epis-monos,sec:synthetic}.
%
In notation and terminology I have stayed close to Bauer's notes as I admire its
readability.

\section{Further reading}\label{sec:further-reading}

Natural candidates for further reading are the aforementioned
notes by Bauer~\cite{Bauer2023} and Streicher~\cite{Streicher2018}, as well as
the standard textbook by van Oosten~\cite{vanOosten2008}.

The categorical semantics of (polymorphic) type theories using realizability is
treated in a variety of works, such as Reus's tutorial paper~\cite{Reus1999},
Streicher's notes~\cite{Streicher2018}, Amadio and Curien's
textbook~\cite{AmadioCurien1998} and Jacobs's textbook~\cite{Jacobs1999}; see
also the references listed on~\cite[p.~193]{vanOosten2008}.

Those looking to learn more about general models of (higher-order) computability
can consult Longley and Normann's comprehensive
textbook~\cite{LongleyNormann2015}.

To those interested in the history of realizability we recommend Troelstra's
proof-theoretic survey~\cite{Troelstra1998} and van Oosten's
essay~\cite{vanOosten2002} for its categorical aspects.

If you are interested in the formalization of mathematics, then you should look
at Chhabra's ongoing \emph{Cubical Agda} development~\cite{Chhabra2023}\footnote{With
  the caveat that the combinatory algebras are assumed to be total---at least for
  the moment.}.


%%% Local Variables:
%%% mode: latexmk
%%% TeX-master: "../main"
%%% End:
