\chapter{Introduction}

\cite{Kleene1945}
\cite{Hyland1982}\cite{HJP1980}\cite{Pitts1981}

\section{Aims}

\section{Exercises}

The exercises are interspersed in the text, but each chapter ends with a list of
its exercises for reference.
%
There are \total{allexercises} exercises in total.

\section{References}

\cite{Bauer2023}\cite{vanOosten2008}

\cite{Streicher2018}

\section{Further reading}

Natural candidates for further reading are the aforementioned
notes by Bauer~\cite{Bauer2023} and Streicher~\cite{Streicher2018}, as well as
the standard textbook by van Oosten~\cite{vanOosten2008}.

The categorical semantics of (polymorphic) type theories using realizability is
treated in a variety of works, such as Reus's tutorial paper~\cite{Reus1999},
Streicher's notes~\cite{Streicher2018}, Amadio and Curien's
textbook~\cite{AmadioCurien1998} and Jacob's textbook~\cite{Jacobs1999}; see
also the references listed on~\cite[p.~193]{vanOosten2008}.

Those looking to learn more about general models of (higher-order) computability
can consult Longley and Normann's comprehensive
textbook~\cite{LongleyNormann2015}.

To those interested in the history of realizability we recommend Troelstra's
proof-theoretic survey~\cite{Troelstra1998} and van Oosten's
essay~\cite{vanOosten2002} for its categorical aspects.

If you are interested in the formalization of mathematics, then you should look
at Chhabra's ongoing \emph{Cubical Agda} development~\cite{Chhabra2023}\footnote{With
  the caveat that the combinatory algebras are assumed to be total---at least for
  the moment.}.


%%% Local Variables:
%%% mode: latexmk
%%% TeX-master: "../main"
%%% End:
